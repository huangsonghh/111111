% IMPORTANT: PLEASE USE XeLaTeX FOR TYPESETTING
\documentclass{sintefbeamer}[dark]
\usepackage{xeCJK}
\usepackage{bm}
\usepackage{mathrsfs}
\usepackage{setspace}
\usepackage{amsmath}
\usepackage{tikz}
\usepackage{graphicx}
% \usepackage{circledtext} % for circled text

            
\AtBeginEnvironment{block}{
    \setstretch{1.35}
    \setlength{\abovedisplayskip}{2pt}
    \setlength{\belowdisplayskip}{2pt}
    \setlength{\abovedisplayshortskip}{2pt}
    \setlength{\belowdisplayshortskip}{2pt}
}
% Define the 'szu' theorem style

% \makeatletter % 可以使其中的@生效
\setbeamertemplate{theorem begin}{
  \begin{\inserttheoremblockenv}
  {%
    \leftskip0pt\upshape \inserttheoremname%
    \ifx\inserttheoremnumber\empty\else\ \inserttheoremnumber\fi
    \ifx\inserttheoremaddition\empty\else\ (\inserttheoremaddition)\fi%
  }%
  \setlength{\parindent}{2em}
  }
\setbeamertemplate{theorem end}{\end{\inserttheoremblockenv}}
\setbeamersize{description width=2em}

% \makeatother

% Custom colors
\setbeamercolor{block title}{bg=thmcolor,fg=white}
\setbeamercolor{block body}{}



\newtheorem*{定义}{定义}
\newtheorem*{例}{例}
\newtheorem*{定理}{定理}
\newtheorem*{证}{证}
\newtheorem*{推论}{推论}
\newtheorem*{性质}{性质}
\newtheorem{引理}{引理}

% meta-data
\title{深圳大学\\数学科学学院}
\subtitle{矩阵理论讲义}
\author{\href{2300201006@email.szu.edu.cn}{黄松}}
\date{\today}
\titlebackground{images/beijing}


% document body
\begin{document}

\maketitle

\section{范数理论}

\begin{frame}{范数理论}{范数}
	
\end{frame}

\begin{frame}{范数理论}{向量范数}
	\begin{定义}[4.1]
		若对于${\forall}\bm{x}\in\mathbf{C}^{n}$都有一个实数$\left \| \bm{x}\right \|$与之对应,且满足:
		\\
		$\textcircled{1}$正定性:当$\bm{x}\neq\bm{0},\left \| \bm{x}\right \|>0$;当$\bm{x}=\bm{0},\left \| \bm{x}\right \|=0;$
		\\
		$\textcircled{2}$齐次性:${\forall}k\in\mathbf{C},\left \| k\bm{x}\right \|=|k|\left \| \bm{x}\right \|;$
		\\
		$\textcircled{3}$三角不等式:${\forall}\bm{x},\bm{y}\in\mathbf{C}^{n}$,都有$\left \| \bm{x}+\bm{y}\right \|\le\left \| \bm{x}\right \|+\left \| \bm{y}\right \|;$
		\linebreak
        则称$\|\bm{x}\|$为$\bm{x}$的向量范数,定义了范数的线性空间称为附范数线性空间.
    \end{定义}
       
        上述$\textcircled{1},\textcircled{2},\textcircled{3}$称为向量范数三公理.
\end{frame}

\begin{frame}{范数理论}{向量范数性质}
	由向量范数的定义可得范数的性质如下:
	\begin{性质}
		\begin{itemize}[<+->]
		\item$\|-\bm{x}\|=\|\bm{x}\|$;
		\item$|\, \|\bm{x}\|-\|\bm{y}\|\,|\le \|\bm{x}\|+\|\bm{y}\|$.
		\end{itemize}
	\end{性质}
\end{frame}

\begin{frame}{范数理论}{向量范数}
	\begin{例}[4.1]
		设$\bm{x}=(x_{1},x_{2},\cdots,x_{n})^{T}\in\mathbf{C}^{n}$,定义
		$$
		\|\bm{x}\|_{1}=\sum_{k=1}^{n}|x_{k}| 
		$$
		\linebreak
		则$\|\bm{x}\|_{1}$是向量$\bm{x}$的一种范数,称为向量1-范数.
		\end{例}
	\pause
	\begin{例}[4.2]
		设$\bm{x}=(x_{1},x_{2},\cdots,x_{n})^{T}\in\mathbf{C}^{n}$,定义
		$$
		\|\bm{x}\|_{2}=\sqrt{\sum_{k=1}^{n}|x_{k}|^{2}}=\sqrt{\sum_{k=1}^{n}\overline{x_{k}}x_{k}}=\sqrt{\bm{x}^{H}\bm{x}}=(\bm{x},\bm{x})^{\frac{1}{2} } 
		$$
		\linebreak
		则$\|\bm{x}\|_{2}$是向量$\bm{x}$的一种范数,称为向量2-范数.
		\end{例}
	\pause
	$\|\bm{x}\|_{2}$也叫Euclid范数,就是通常意义下的向量的长度.
	\end{frame}

\begin{frame}{范数理论}{向量范数}
	向量的2-范数有如下重要的性质
	\begin{性质}
		对${\forall}\bm{x}\in\mathbf{C}^{n}$和任意的酉矩阵$\bm{U}$,有
		$$
		\|\bm{U}\bm{x}\|_{2}= \sqrt{(\bm{U}\bm{x})^{H}(\bm{U}\bm{x})} =\sqrt{\bm{x}^{H}\bm{U}^{H}\bm{U}\bm{x}}=\sqrt{\bm{x}^{H}\bm{x}}=\|\bm{x}\|_{2}
		$$
		\linebreak
		这一性质称为向量2-范数的酉不变性.
		\end{性质}
\end{frame}


\begin{frame}{范数理论}{向量范数}
	\begin{例}[4.3]
		设$\bm{x}=(x_{1},x_{2},\cdots,x_{n})^{T}\in\mathbf{C}^{n}$,定义
		$$
		\|\bm{x}\|_{\infty }=\underset{k}{max}|x_{k}|
		$$
		\end{例}
	\pause
	\begin{证}
		$\textcircled{1},\textcircled{2}$的正定性,齐次性显然.
		\\
		以下证$\textcircled{3}$三角不等式:
		\\
		设$\bm{y}=(y_{1},y_{2},\cdots,y_{n})^{T}\in\mathbf{C}^{n}$,有
		$$
		\|\bm{x}+\bm{y}\|_{\infty}=\underset{k}{max}|x_{k}|+\underset{k}{max}|y_{k}|=\|\bm{x}\|_{\infty}+\|\bm{y}\|_{\infty}
		$$
		\linebreak
		故$\|\bm{x}\|_{\infty}$是$\mathbf{C}^{n}$的一种向量范数.
		\end{证}
\end{frame}

\begin{frame}{范数理论}{Young不等式}
	\begin{引理}[Young不等式]
		对任意实数$\alpha \ge 0$和$\beta \ge 0$,都有
		$$
		\alpha \beta \le \frac{\alpha ^{p} }{p}+\frac{\beta ^{q} }{q}  
		$$
		\linebreak
		其中$p>1,q>1$,且$\frac{1}{p} +\frac{1}{q} =1$,p,q称为共轭指数.
		
		\end{引理}
\end{frame}

\begin{frame}{范数理论}{H$\ddot{o}$lder不等式}
	\begin{定理}[H$\ddot{o}$lder不等式]
		对任意$x_{k},y_{k}\in\mathbf{C},k=1,2,\cdots,n$有
		$$
	    \sum_{k=1}^{n}|x_{k}|\,|y_{k}|\le (\sum_{k=1}^{n}|x_{k}|^{p} )^{\frac{1}{p}}(\sum_{k=1}^{n}|y_{k}|^{q} )^{\frac{1}{q} } 
		$$
		\linebreak
		其中,$p>1,q>1$,且$\frac{1}{p}+\frac{1}{q} =1$.
		\pause
		当$p=q=2$,就是有名的Cauchy-Schwarz不等式.
		\end{定理}
\end{frame}

\begin{frame}{范数理论}{Minkowski不等式}
	\begin{定理}[Minkowski不等式]
		对任何$p\ge1 $,有
		$$
		(\sum_{k=1}^{n}|x_{k}+y_{k}|^{p} )^{\frac{1}{p} }\le (\sum_{k=1}^{n}|x_{k}|^{p} )^{\frac{1}{p} }+ (\sum_{k=1}^{n}|y_{k}|^{p} )^{\frac{1}{p} }
		$$
		\end{定理}
	\pause
	\begin{例}[4.4]
		设$\bm{x}=(x_{1},x_{2},\cdots,x_{n})^{T}\in\mathbf{C}^{n}$,定义
		$$
		\|\bm{x}\|_{p}=(\sum_{k=1}^{n}|x_{k}|^{p})^{\frac{1}{p}}\qquad 1\le p<+\infty
		$$
		\linebreak
		则$\|\bm{x}\|_{p}$是$\mathbf{C}^{n}$上的一种向量范数.
		\end{例}
	\end{frame}

\begin{frame}{范数理论}{向量范数}
	\qquad
	对于$\|\bm{x}\|_{p}$,$p=1$即为向量的1-范数,$p=2$即为向量的2-范数,$p=+\infty$是否为$\infty$-范数呢,有如下定理.
	\begin{定理}[4.3]
		设$\bm{x}=(x_{1},x_{2},\cdots,x_{n})^{T}\in\mathbf{C}^{n}$,则
		$$
		\lim_{p \to +\infty}\|\bm{x}\|_{p}=\|\bm{x}\|_{\infty}
		$$
		\end{定理}
\end{frame}

	\begin{frame}{范数理论}{向量范数的构造}
		\begin{定理}[4.4]
		设$\mathbf{A}\in\mathbf{C}_{n}^{m\times n}$,$\,\|\bullet\|_{a} $是$\mathbf{C}^{m}$上的一种向量范数,对$\forall\bm{x}\in\mathbf{C}^{n}$,定义
		$$
		\|\bm{x}\|_{b}=\|\bm{A}\bm{x}\|_{a}
		$$
		\linebreak
		则$\|\bm{x}\|_{b}$是$\mathbf{C}^{n}$中的向量范数.
		\end{定理}
	\pause
	由此可见,由一个已知的向量范数可构造出无穷多的新的向量范数.
	\end{frame}

\begin{frame}{范数理论}{向量范数}
	\begin{例}[4.5]
		设$\bm{A}$是$n$阶Hermite正定矩阵,对任意$\bm{x}\in\mathbf{C}^{n}$,定义
		$$
		\|\bm{x}\|_{\bm{A}}=\sqrt{\bm{x}^{H}\bm{A}\bm{x}}
		$$
		\linebreak
		则$\|\bm{x}\|_{\bm{A}}$是一种向量范数.
		\end{例}
	\pause
	证: 因为A是Hermit正定矩阵,故存在n阶非奇异矩阵$Q$,使$A=Q^{H}Q$,于是
	\begin{equation*}
	    \begin{aligned}
		&\parallel x\parallel_A=\sqrt{x^{\mathrm{H}}Q^{\mathrm{H}}Qx}=\sqrt{(Qx)^{\mathrm{H}}(Qx)}=\parallel Qx\parallel_2
		\end{aligned}
	\end{equation*}
	\end{frame}
		
	\begin{frame}{范数理论}{向量范数}
	\begin{例}[4.6]
		设$\bm{V}_{n}(\mathbf{C})$是复数域$\mathbf{C}$上的$n$维线性空间,$\bm{\varepsilon _{1}},\bm{\varepsilon _{2}},\cdots,\bm{\varepsilon _{n}}$是$\bm{V}_{n}(\mathbf{C})$的一组基.$\forall\bm{\alpha} \in\bm{V}_{n}(\mathbf{C})$可唯一地表示为$\bm{\alpha} =\sum_{k=1}^{n} x_{k}\bm{\varepsilon _{k}},\bm{x}=(x_{1}, x_{2},\cdots,x_{n})^{T}\in\mathbf{C}^{n}$.又设$\|\bullet\|$是$\mathbf{C}^{n}$上的向量范数,定义
		$$
		\|\bm{\alpha}\|_{v}=\|\bm{x}\|
		$$
		\linebreak
		则$\|\bm{\alpha}\|_{\nu}$是$\bm{V}_{n}(\mathbf{C})$上的向量范数.
		\end{例}
\end{frame}

\begin{frame}{范数理论}{向量范数}
	\begin{例}[4.7]
		设$f(x)\in\mathbf{C}[a,b]$定义
		$$
		\|f(t)\|_{1}=\int_{a}^{b}|f(t)|dt
		$$
		$$
		\|f(t)\|_{p}=(\int_{a}^{b}|f(t)|^{p}dt)^{\frac{1}{p} }
		$$
		$$
		\|f(t)\|_{\infty}=\max_{a\le t\le b}|f(t)| 
		$$
		\linebreak
		则它们都是$\mathbf{C}[a,b]$上的向量范数.
		\end{例}
	\end{frame}
		
	\begin{frame}{范数理论}{向量范数}
		积分形式的H$\ddot{o}$lder不等式(其中$p$,$q$为共轭指数)
		\begin{equation*}
			\begin{aligned}
				&\int_a^b\big|\:f(t)\:g(t)\:\big|\:\mathrm{d}t\leqslant(\int_a^b\big|\:f(t)\:\big|^p\:\mathrm{d}t)^{\frac{1}{p}}\big(\int_a^b\big|\:g(t)\:\big|^q\:\mathrm{d}t\big)^{\frac{1}{q}}
			\end{aligned}
		\end{equation*} 
		Minkowski不等式
		\begin{equation*}
			\begin{aligned}
				&\int_{a}^{b}(\begin{array}{c|c|c}f(t)+g(t)&^{p})\end{array}^{\frac{1}{p}}\leqslant(\int_{a}^{b}\bigg|\:f(t)\:\bigg|^{p}\:\mathrm{d}t)^{\frac{1}{p}}+(\int_{a}^{b}\bigg|\:g(t)\:\bigg|^{p}\:\mathrm{d}t)^{\frac{1}{p}}\quad p\geqslant1
			\end{aligned}
		\end{equation*} 
\end{frame}

\begin{frame}
	\begin{例}[4.8]
		设$\boldsymbol{x}=(x_1,x_2,\cdots,x_n)^{\mathrm{T}}\in\mathbf{C}^n$,定义
		\begin{equation*}
			\begin{aligned}
				&\parallel x\parallel_p=(\sum_{k=1}^n\mid x_k\mid^p)^{\frac{1}{p}}\quad0<p<1
			\end{aligned}
		\end{equation*}
		由于它不满足定义4.1中的$\textcircled{3}$,故它不是$\mathbf{C}^n$上的向量范数。
	\end{例}
		例如在$\mathbf{R}^n$中,$x=(1,0,\cdots,0)^{\mathrm{T}},y=(0,1,0,\cdots,0)^{\mathrm{T}},p=\frac{1}{2}$,则
        \begin{equation*}
			\begin{aligned}
				&\parallel x+y\parallel_{\frac12}=4,\parallel x\parallel_{\frac12}=1,\parallel y\parallel_{\frac12}=1
			\end{aligned}
		\end{equation*}
		故$\parallel x\parallel_{\frac12}$不是$\mathbf{R}^n$上的向量范数。
\end{frame}

\begin{frame}
	\begin{定义}[4.2]
		设$\|\bullet\|_{a}$和$\|\bullet\|_{b}$是$\mathbf{C}^{n}$上的两种向量范数,如果存在正数$c_{1},c_{2}$,使对任意$\bm{x}\in\mathbf{C}^{n}$都有
		$$
		c_{1}\|\bm{x}\|_{b}\le\|\bm{x}_{a}\|\le c_{2}\|\bm{x}\|_{b}
		$$
		\linebreak
		则称向量范数$\|\bullet\|_{a}$与$\|\bullet\|_{b}$等价.
		\end{定义}
\end{frame}

\begin{frame}
	\begin{定理}[4.5]
		设$\|\bullet\|$是$\bm{V}_{n}(\bm{F})$上的任一向量范数,$\bm{\varepsilon _{1}},\bm{\varepsilon _{2}},\cdots,\bm{\varepsilon _{n}}$为$\bm{V}_{n}(\bm{F})$的一组基.$\forall\bm{\alpha} \in\bm{V}_{n}(\bm{F})$可唯一表示成$\bm{\alpha} =\sum\limits_{k=1}^{n} x_{k}\bm{\varepsilon _{k}},\bm{x}=(x_{1}, x_{2},\cdots,x_{n})^{T}\in\bm{F}^{n}$,则$\|\bm{\alpha}\|$是$x_{1},x_{2},\cdots,x_{n}$的连续函数.
		\end{定理}
	\pause
	\begin{定理}[4.6]
		$n$维线性空间$\bm{V}_{n}(\bm{F})$上的任意两个向量范数都是等价的.
		\end{定理}
\end{frame}
\begin{frame}
	\begin{定义}[4.3]
		设$\{\bm{x}^{(k)}\}$是$\mathbf{C}^{n}$中的向量序列,其中$\bm{x}^{(k)}=(x_{1}^{(k)},\cdots,x_{i}^{(k)},\cdots,x_{n}^{(k)})^{T}$,若$\underset{k \to +\infty}{lim} x_{i} ^{(k)}=x_{i}(i=1,2,\cdots,n)$,则称向量序列$\{\bm{x}^{(k)}\}$是收敛的,并说$\{\bm{x}^{(k)}\}$的极限为向量$\bm{x}=(x_{1},\cdots,x_{i},\cdots,x_{n})^{T}$,记为
		$$
		\lim_{k \to +\infty} \bm{x}^{(k)}=\bm{x}
		$$
		\linebreak
		向量序列不收敛时称为发散的.
		\end{定义}
\end{frame}
\begin{frame}
	\qquad
	与数列收敛相类似,很容易证明向量序列的收敛性具有以下性质:
	\begin{性质}
		设$\{\bm{x}^{(k)}\},\{\bm{y}^{(k)}\}$是$\mathbf{C}^{n}$中的两个向量序列,$\lambda ,\mu $是两个复常数,$\bm{A}\in\mathbf{C}^{m\times n}$,且$\underset{k\to+\infty}{lim}{\{\bm{x}^{(k)}\}}=\bm{x},\underset{k\to+\infty}{lim}{\{\bm{y}^{(k)}\}}=\bm{y}$,则:
		\par
		$\textcircled{1}\underset{k \to +\infty}{lim} (\lambda\bm{x}^{(k)}+\mu\bm{y}^{(k)})=\lambda\bm{x}+\mu\bm{y}$;
		\par
		$\textcircled{2}\underset{k \to +\infty}{lim}\bm{A}\bm{x}^{(k)}=\bm{A}\bm{x}$.
		
		\end{性质}
	\pause
	\begin{定理}[4.7]
		$\mathbf{C}^{n}$中向量序列$\{\bm{x}^{(k)}\}$收敛于向量$\bm{x}$的充分必要条件是,对于$\mathbf{C}^{n}$上任一向量范数$\|\bullet\|$,都有
		$$
		\underset{k\to \infty}{lim}\|\bm{x}^{(k)}-\bm{x}\|=0
		$$
		\end{定理}
	
\end{frame}

\section{矩阵范数}
\begin{frame}
在许多场合需要度量矩阵的“大小”,比如矩阵序列的收敛、解线性方程组时的误差分析。一个自然的想法就是将矩阵“拉直”,将其视作一个$mxn$维的向量。\\
若$A\in\mathbf{R}^{m\times n}$,则可将$A$看作$\mathbf{R}^{m\times n}$的向量,可以按照向量范数三公理来定义它的向量范数
\begin{equation*}
	\begin{aligned}
		\parallel\boldsymbol{A}\parallel_{v_1}&=\sum_{i=1}^m\sum_{j=1}^n\mid a_{ij}\mid\\
		\parallel\boldsymbol{A}\parallel_{v_p}&=(\sum_{i=1}^n\sum_{j=1}^n\mid a_{ij}\mid^{\frac1p})^{\frac1p}\quad1<p<+\infty\\
		\parallel\boldsymbol{A}\parallel_{v_\infty}&=\max_{i,j}\mid a_{ij}\mid
	\end{aligned}
\end{equation*}
\end{frame}

\begin{frame}
	\begin{定义}[4.4]
	若$\forall A\in\mathbb{C}^{n\times n}$都有一个实数 $\parallel A\parallel$与之对应,且满足:
	\begin{equation*}
		\begin{aligned}
			&\text{ 正定性:当 }A\neq\boldsymbol{O},\parallel A\parallel>0;\text{当 }A=\boldsymbol{O}\text{时,}\parallel A\parallel=0;\\
			&\text{ 齐次性: }\forall k\in\mathbf{C},\parallel k\mathbf{A}\parallel=\mid k\mid\parallel\mathbf{A}\parallel;\\
			&\text{三角不等式:}\forall A,B\in\mathbb{C}^{n\times n},\text{都有}\parallel A+B\parallel\leqslant\|A\parallel+\|B\parallel;\\
			&\text{相容性:}\forall A,B\in\mathbb{C}^{n\times n},\text{都有}\parallel AB\parallel\leqslant\|A\parallel\|B\parallel.
		\end{aligned}
	\end{equation*}
	则称$\left\|A\right\|$为$ C^{n\times n}$上矩阵$A$的矩阵范数.\\
\end{定义}
	\pause
	\textbf{性质1} \qquad $\parallel-A\parallel=\parallel A\parallel.$\\
	\textbf{性质2} \qquad $\mid\lVert A\rVert-\lVert B\rVert\mid\leqslant\lVert A-B\rVert.$
\end{frame}

\begin{frame}
	\begin{例}[4.9]
		设$\mathbf{A}=\left(a_{ij}\right)_{n\times n}\in\mathbf{C}^{n\times n}$,定义
		\begin{equation*}
			\begin{aligned}
			&\parallel A\parallel_{m_1}=\sum_{i=1}^n\sum_{j=1}^n\mid a_{ij}\mid 
		\end{aligned}
		\end{equation*}
		则$\parallel A\parallel_{m_1}$是$\mathbb{C}^{n\times n}$上的一种矩阵范数,称为矩阵的$m_1-$范数。
	\end{例}
\end{frame}

\begin{frame}
	\begin{例}[4.10]
		设$\boldsymbol{A}=\left(a_{ij}\right)_{n\times n}\in\mathbf{C}^{n\times n}$,定义
		\begin{equation*}
		\begin{aligned}
			&\parallel A\parallel_F=\parallel A\parallel_{m_2}=\sqrt{\sum_{i=1}^n\sum_{j=1}^n\mid a_{ij}\mid^2}=(\sum_{i=1}^n\sum_{j=1}^n\mid a_{ij}\mid^2)^{\frac{1}{2}}=\sqrt{\operatorname{tr}(A^{\text{H}}A)}
		\end{aligned}
		\end{equation*}
		则$\parallel A\parallel_F$是$\mathbb{C}^{n\times n}$,上的一种矩阵范数,称为矩阵的\textbf{Frobenius}范数,简称$F-$范数。
	\end{例}
\end{frame}

\begin{frame}
	\begin{定理}[4.8]
		设$A\in\mathbb{C}^{n\times n}$,记$A=(\boldsymbol{\alpha}_1,\cdots,\boldsymbol{\alpha}_j,\cdots,\boldsymbol{\alpha}_n),\boldsymbol{\alpha}_j\in\mathbf{C}^n,j=1,2,\cdots,n$,则:
		\begin{equation*}
			\begin{aligned}
				&\textcircled{1}\parallel A\parallel_F^2=\sum_{j=1}^n\parallel\alpha_j\parallel_2^2;\\
				&\textcircled{2}\parallel\mathbf{A}\parallel_F^2=\mathrm{tr}(\mathbf{A}^\mathrm{H}\mathbf{A})=\sum_{i=1}^n\lambda_i(\mathbf{A}^\mathrm{H}\mathbf{A});
                \text{其中 }\lambda_i(A^\text{н}A)(i=1,2,\cdotp\cdotp\cdotp,n)\text{ 表示 }A^\text{н}A\text{ 的第i 个特征值}.\\
				&\textcircled{3}\text{对任意的}n\text{阶酉阵}U,V\text{有}
				\parallel U\text{A }\parallel_F=\parallel AV\parallel_F=\parallel UAV\parallel_F=\parallel U^\text{H}AV\parallel_F=\parallel A\parallel_F.
				\text{称之为F- 范数的酉不变性}.
			\end{aligned}
		\end{equation*}
    \end{定理}
\end{frame}

\begin{frame}
	\begin{例}[4.11]
		设$\mathbf{A}=\left(a_{ij}\right)_{n\times n}\in\mathbf{C}^{n\times n}$,定义
		\begin{equation*}
			\begin{aligned}
				&\parallel A\parallel_{m_\infty}=n\max_{i,j}\mid a_{ij}\mid 
			\end{aligned}
		\end{equation*}
		则$\parallel\mathbf{A}\parallel_{m_\infty}$是$\mathbb{C}^{n\times n}$上的矩阵范数,称为矩阵的$m_{\infty}-$范数。
	\end{例}
\end{frame}

\begin{frame}
	\begin{定理}[4.9]
		$\parallel\cdot\parallel_{m}$是$\mathbb{C}^{n\times n}$上的矩阵范数,$\mathbf{A}=\left(a_{ij}\right)_{n\times n}\in\mathbf{C}^{n\times n}$,则:
		\begin{equation*}
			\begin{aligned}
				&\textcircled{1}\parallel\mathbf{A}\parallel\text{ 是 }a_{ij}\text{ 的连续函数 },i=1,\cdots,n;j=1,\cdots,n;\\
				&\textcircled{1}\mathbf{C}^{n\times n}\text{ 上任意两个矩阵范数等价}.
			\end{aligned}
		\end{equation*}
	\end{定理}
\end{frame}

\begin{frame}
	\begin{定义}[4.12]
		设$\parallel\cdot\parallel_m$是$\mathbb{C}^{n\times n}$上的矩阵范数,$\parallel\cdot\parallel_v$是$\mathbb{C}^{n}$上的向量范数,若$\forall A\in\mathbb{C}^{n\times n}$和$\forall x\in\mathbb{C}^n$都有
		\begin{equation*}
			\begin{aligned}
				&\parallel Ax\parallel_v\leqslant\parallel A\parallel_m\parallel x\parallel_v
			\end{aligned}
		\end{equation*}
		则称矩阵范数$\parallel\cdot\parallel_m$与向量范数$\parallel\cdot\parallel_v$是相容的。
	\end{定义}
\end{frame}

\begin{frame}
	\begin{例}[4.12]
      求证$\mathbb{C}^{n\times n}$上的矩阵$m_{1}-$范数与$\mathbb{C}^{n}$上向量的1-范数相容。
	\end{例}
	\begin{例}[4.13]
       求证$\mathbb{C}^{n\times n}$的矩阵$F-$范数与$\mathbb{C}^{n}$上向量的2-范数相容。
	\end{例}
	\begin{例}[4.14]
      求证$\mathbb{C}^{n\times n}$上矩阵的$m_{\infty}-$范数与$\mathbb{C}^{n}$上向量的$\infty-$范数相容。
	\end{例}
\end{frame}

\begin{frame}
	\begin{定理}[4.10]
		设$\parallel\cdot\parallel_m$是$\mathbb{C}^{n\times n}$上的一种矩阵范数,则在$\mathbb{C}^{n}$上必存在与它相容的向量范数。
	\end{定理}
\end{frame}

\backmatter
\end{document}
